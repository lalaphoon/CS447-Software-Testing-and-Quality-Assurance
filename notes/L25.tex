\documentclass[11pt]{article}
\usepackage{url}
\usepackage{listings}
\usepackage{tikz}
\usepackage{fontspec}
\usepackage{enumitem}
\setmainfont{Latin Modern Roman}
\usetikzlibrary{arrows,automata,shapes}
\tikzstyle{block} = [rectangle, draw, fill=blue!20, 
    text width=5em, text centered, rounded corners, minimum height=2em]
\tikzstyle{bt} = [rectangle, draw, fill=blue!20, 
    text width=1em, text centered, rounded corners, minimum height=2em]

\newtheorem{defn}{Definition}
\newtheorem{crit}{Criterion}
\newcommand{\true}{\mbox{\sf true}}
\newcommand{\false}{\mbox{\sf false}}

\newcommand{\handout}[5]{
  \noindent
  \begin{center}
  \framebox{
    \vbox{
      \hbox to 5.78in { {\bf Software Testing, Quality Assurance and Maintenance } \hfill #2 }
      \vspace{4mm}
      \hbox to 5.78in { {\Large \hfill #5  \hfill} }
      \vspace{2mm}
      \hbox to 5.78in { {\em #3 \hfill #4} }
    }
  }
  \end{center}
  \vspace*{4mm}
}

\newcommand{\lecture}[4]{\handout{#1}{#2}{#3}{#4}{Lecture #1}}
\topmargin 0pt
\advance \topmargin by -\headheight
\advance \topmargin by -\headsep
\textheight 8.9in
\oddsidemargin 0pt
\evensidemargin \oddsidemargin
\marginparwidth 0.5in
\textwidth 6.5in

\parindent 0in
\parskip 1.5ex
%\renewcommand{\baselinestretch}{1.25}

\usepackage[listings]{tcolorbox}
\newtcbinputlisting{\codelisting}[3][]{
    extrude left by=1em,
    extrude right by=2em,
    listing file={#3},
    fonttitle=\bfseries,
    listing options={basicstyle=\ttfamily\footnotesize,numbers=left,language=Java,#1},
    listing only,
    hbox,
}

\begin{document}

\lecture{25 --- March 20, 2019}{Winter 2019}{Patrick Lam}{version 1}

\section*{More on Flaky Tests}

Google also has a blog post about their experience with flaky tests. It provides a different
perspective on flaky tests.

\url{https://testing.googleblog.com/2016/05/flaky-tests-at-google-and-how-we.html}

Here are some useful facts from that post:

\begin{itemize}[noitemsep]
\item Rate of flakiness: 1.5\% of all test runs. Over time, their
  flaky test insertion rate is about the same as the flaky test
  removal rate. The number of \emph{tests} which are occasionally
  flaky is 16\%.
\item Code can only be submitted after it passes tests. There are also
  pre-release test suites which must succeed before the project can be
  released. This is presumably after correcting for flakiness.
\item The vast majority (84\%) of post-submit test failures in continuous
  integration are associated with flaky tests. 
\item The flaky test rate means that if you have 1000 pre-release tests,
  you should expect 15 failing tests which require manual investigation.
  If you ignore them, you might well be missing an actual problem.
\item Mitigation: in addition to what I mentioned last time (automatic
  re-running), Google also allows re-running only flaky tests.  Known
  flaky tests get automatically re-run 3 times before reporting a
  fail. This slows down notification of test failures, too. Also, tests
  that are seen to be flaky are automatically quarantined (not run every time;
  bug reported). 
\end{itemize}

\section*{Case Study: Mocks/Stubs}

Next, we'll summarize another Google case study, this time about mocks/stubs.

\url{https://testing.googleblog.com/2016/11/what-test-engineers-do-at-google.html}

This blog entry describes the work of a Google Test Engineer in improving test infrastructure.

\paragraph{Situation.} Legacy system. Flaky tests: large and brittle. End-to-end tests
were difficult to introduce fakes into. System used many external dependencies.

\paragraph{First false start.} Splitting the end-to-end tests. Didn't work: would have needed
refactoring for entire legacy system, not just the part the author's team was working on.

\paragraph{Second false start.} Mock services that were not actually required. Not viable: 
dependencies changed often. Imposed test maintenance cost.

\paragraph{Actual solution.} Replace the client code (which calls depended-on services)
by unit tests of calls to RPC stubs. Stub implemented with mock objects.
Then, in another test, send the data to the actual service (i.e. test the tests).

The benefits are much faster tests (10$\times$ speedup: from 30 minutes to 3 minutes)
which are not at all flaky and which can run on developer machines.

\section*{Airbnb Testing Infrastructure}

Let's change it up here and talk about Airbnb instead.

\url{http://nerds.airbnb.com/testing-at-airbnb/}

This describes how the author worked with colleagues to introduce a
(developer-based) testing culture at Airbnb.

\paragraph{Running tests locally.} As I mentioned earlier, running tests
in prod-like environments helps a lot. Airbnb did so using Vagrant boxes 
(``ready to run tests out of the box.'') Airbnb runs on Ruby on Rails and 
they use Zeus to allow developers to start up the Rails environment super
quickly.

\paragraph{Continuous Integration.} They also use Solano for Continuous
Integration (as described in the L24 notes). Tests run at Airbnb (not
in the public cloud), in parallel for improved throughput.

Furthermore, their Github displays the build/test status of every pull
request, and presumably developers only merge pull requests that pass
the tests.

\end{document}
