%%%%%%%%%%%%%%%%%%%%%%%%%%%%%%%%%%%%%%%%%%%%%%%%%%%%%%%%%%%%%%%%%%%%%%%%%%%%%%%
%% About this LaTeX file:
%%
%% This is a sample LaTeX file for a UWaterloo exam/test document using the 
%% Odyssey exam management system and the Crowdmark online grading system.
%% Both Odyssey and Crowdmark cover over parts of every test page.
%% This LaTeX file sets up a page layout with this in mind.

%% Look for this LaTeX file on the University of Waterloo's
%% help page for the Crowdmark system: https://uwaterloo.ca/crowdmark/
%% or https://uwaterloo.ca/crowdmark/midterms-and-final-exams.
%% Sample pdf files showing the page layout options are there too, as are
%% documents describing how to use the Odyssey and Crowdmark systems for
%% managing and grading (online grading) your tests, quizzes, and exams.

%% Why the page layout is the way it is:
%%
%% Odyssey uses the bottom .65 inches of every page for a page number
%% and about 4 inches at the top of the cover page for exam information.
%% Crowdmark uses the top 1.5 inches of every page (including the cover page)
%% for QR coded booklet/page information.
%% When Odyssey and Crowdmark are used together, the cover page starts
%% with a 1.5 inch QR code area followed by the Odyssey 4 inch exam area.

%% In fact, Odyssey can use as little as 3.75 inches when there are no special
%% materials listed for an exam.  And, it can use more than 4 inches when there
%% are many listed materials.

%% How to use this LaTeX file:
%%
%% This sample LaTeX file can be used for 4 variations of page layout.
%% Two variations are for Crowdmark, when the LaTeX variable tmargin is 
%% set to 1.5in (default):
%%   * Odyssey and Crowdmark: use the file as is
%%     (the cover page framed box with exam info is covered up by Odyssey)
%%   * Crowdmark without Odyssey: use the file as is 
%%     (the framed box is not covered up)
%%
%% And two variations that do not use Crowdmark, when the LaTeX 
%% variable tmargin is set to .25in (the larger Crowdmark top margin 
%% of 1.5 inches is no longer needed):
%%   * Odyssey without Crowdmark
%%   * Without Odyssey or Crowdmark

% Version 1 (Feb 19, 2019), Paul Kates (pkates@uwaterloo.ca).
%%%%%%%%%%%%%%%%%%%%%%%%%%%%%%%%%%%%%%%%%%%%%%%%%%%%%%%%%%%%%%%%%%%%%%%%%%%%%%%

\documentclass[12pt]{article}

\usepackage[utf8]{inputenc}
% Use showframe to see layout boundaries during drafts, but not for a printed
% exam.
%\usepackage{showframe}

%% Page layout
\usepackage{geometry}  % used for page layout
  % Size numbers you can change:
  % hmargin{leftside,rightside} page margins can be adjusted to your liking.
  % The page looks symmetric with the left, right values of .5in and .68in.
  %
  % Size numbers not meant for change.  Sizes are determined by the heights of
  % the areas covered by Odyssey and Crowdmark.
  % tmargin = margin along the top of every page 
  %         = 1.5 inches when used with Crowdmark, or
  %         = .25 inches when used without Crowdmark
  % bmargin = margin along the bottom of every page = .65 inches; from Odyssey
  % \myodysseyheight = height of this file's cover page exam info area
  %                    which is meant to fit inside (be covered up by)
  %                    the Odyssey exam info area (of about 4in in height
  \newlength{\myodysseyheight} 
  \setlength{\myodysseyheight}{3.6in}  % typical range [3.6,4] inches

  \geometry{letterpaper, % 8.5 x 11 inch page (legalpaper 8.5x14in also works)
          tmargin=1.5in, % page top margin when using Crowdmark
         %tmargin=.25in, % page top margin when not using Crowdmark
          hmargin={.5in,.68in}, % leftside, rightside page margins 
                                % (page looks symmetric)
          bmargin=.65in, % page bottom margin
          includehead    % place header in body of text below Crowdmark QR
          }

%% Footer and header
% The footer below will be covered up by Odyssey's footer.  But, when not using
% Odyssey, the footer will show the same information as Odyssey: exam title and
% page number.
\usepackage{lastpage}  % for page number of last page
\usepackage{fancyhdr}  % for setting footer
  \pagestyle{fancy}
  \fancyhead{}         % turn off default header and footer 
  \fancyfoot{}
  \fancyfoot[L]{University of Waterloo}  % left, centre, right footers
  \fancyfoot[C]{SE 465 Midterm Winter 2019}
  \fancyfoot[R]{Page \thepage\ of \pageref{LastPage}}

  \renewcommand{\headrulewidth}{0pt}     % really turn off header rule
  \renewcommand{\footrulewidth}{0.4pt}   % default is 0pt

%% Other LaTeX packages and settings you use can go here:
%
  \usepackage{mathtools, amssymb} % mathtools includes amsmath package
  \usepackage{enumitem}
  \usepackage{listings}
  \usepackage{url}
  \lstset{basicstyle=\footnotesize\ttfamily,breaklines=true}

%% Question grade point values in left margin
\reversemarginpar  % put margin note/grade on left, default is rightside of page
\setlength{\marginparsep}{-.4in}  % default 10pt

%%%%%%%%%%%%%%%%%%%%%%%%%%%%%%%%%%%%%%%%%%%%%%%%%%%%%%%%%%%%%%%%%%%%%%%%%%%%%%%
%% Cover page of test:

\begin{document}
% Save original paragraph indentation size in case you want to restore it.
\newlength{\myoldparindent}
\setlength{\myoldparindent}{\parindent}
\setlength{\parindent}{0em}  % turn off paragraph indentation (set length to 0)
% Use following line later if ever want to restore \parindent:
%\setlength{\parindent}{\myoldparindent} 
 
% A framed box is placed around the area where Odyssey puts its cover page info.
% Anything put into this box will be covered up by Odyssey.
% You can put your exam info here for drafts.
% Or, use this area for your exam cover page if you decide not to use Odyssey.
% If you find that the box frame peeks below the Odyssey cover page info then
% either remove the framing lines by changing \fbox to \mbox below, or make 
% a small reduction in the size of value \myodysseyheight (set above).

\fbox{  % to remove the frame lines, change this \fbox to \mbox
% Start of a minipage container (inside the fbox) for 2 inner minipages below.
\begin{minipage}[t][\myodysseyheight]{\textwidth} 

% Some layout comments you can print on a draft cover page.
\begin{center}
%\large{\textbf{Space above this box is for a Crowdmark QR code}}\\[1ex]
%\large{\textbf{This boxed area will be covered up by Odyssey}}\\[2ex]

% Exam title information.
%
SE 465 Midterm Examination\\
University of Waterloo\\
Term: Winter \hspace{1cm} Year: 2019\\
\end{center}
% Required UWaterloo exam details for cover page:
\begin{minipage}[t]{3.5in} % half of the default 7 in page text width
Date: Thursday, February 28, 2019\\
Time: 18:30 – 20:00 (90 minutes)\\
Instructors: Patrick Lam\\
Lecture Section: 001\\
Exam Type: Open book, open notes, calculators with no communications capabilities\\
Number of exam pages (includes cover page): 8\\
\end{minipage}  % end of first inner minipage of cover page exam details
\hfill%
%
% Student information area:
\begin{minipage}[t]{3.5in} % half of the page text width
\textit{Please Print}\\[1mm]
Last Name \hrulefill\\[2mm]
First Name \hrulefill\\[2mm]
UWaterloo ID \# \hrulefill\\[2mm]
Username \hrulefill\\[2mm]
\end{minipage} % end of second inner minipage of cover page student details
\end{minipage} % end of framed box container minipage
}

%% "Instructions to students" area of the cover page.
%% Every item here is optional.  Even the grading box is here only as 
%% an example.
%
% Adjust the vertical space here if the Odyssey exam info area grows larger
% and starts to cover up the grading box below.
%\vspace{1in}   % height can be 0 to 1 inch to nicely position the grading box
% Grading box.  Use the "Score" row for student scores if not using Crowdmark.
\begin{center}
 \begin{tabular}{|l| c c c c c  ||r|} \hline
 Question & 1 & 2 & 3 & 4 & 5 & Total \\ \hline
 Points & 15 & 10 & 15 & 10 & 10 & 60 \\ \hline
 %Score  &    &    &    &    &    &    &    &    &    &    &    \\ \hline
 \end{tabular}
\end{center}
%
\textbf{Instructions}
\begin{enumerate}
   \item Turn off all communication devices. Communication devices must be stored with your personal items for the duration of the exam. Taking a communication device to a washroom break during this examination is not allowed and will be considered an academic offence.
    \item I shuffled the order of the questions from what I said in class for better page breaks.
    \item The exam lasts \textbf{90} minutes and there are 60 marks.
    \item Verify that your name and student ID number is on the cover page.
    \item If you feel like you need to ask a question, know that the most likely answer is ``Read the Question''. No questions are permitted. If you find that a question requires clarification, proceed by clearly stating any reasonable assumptions necessary to complete the question. If your assumptions are reasonable, they will be taken into account during grading. 
\item Answer the questions in the spaces provided.  If you require 
additional space to answer a question, please use the second last page 
and refer to this page in your solutions. You may tear off the last page 
to use for rough work.
\item Do not write on the Crowdmark QR code at the top of each page.
\item Use a dark pencil or pen for your work.
% \item More instructions, about calculators, formula sheets, asking 
% questions, etc.
\end{enumerate}

% Remove these 2 LaTeX commands when making your own cover page.

%%%%%%%%%%%%%%%%%%%%%%%%%%%%%%%%%%%%%%%%%%%%%%%%%%%%%%%%%%%%%%%%%%%%%%%%%%%%%%%
%% Second page of test, for exam questions.
%
\newpage
\renewcommand{\headrulewidth}{0.4pt}  % put header rule on all non-cover pages 



\section{Mutation Testing [10 Marks]}
Consider this function {\tt Location.add()}.  Write down a (non-trivial, non-stillborn, non-equivalent) mutant to
the {\tt Location.add()} function (indicating your change) as well as a test case that kills
the mutant. (Use JUnit-like syntax; we don't care about the details).
Provide the expected and actual output for your testcase with the
original function as well as for the mutant.

\vspace*{1em}

Assume that you can create a {\tt Location}
object with an expression like {\tt new Location(0.0, 4.65, 3.50, W);}. Describe a {\tt Location} with a tuple like $\langle 0.0, 4.65, 3.50, W\rangle$.
\begin{lstlisting}[language=Java]
    public Location(double x, double y, double z, Object world) {
        this.x = x; this.y = y; this.z = z; this.world = world;
    }
  
    // credit: org.bukkit.Location.java
    /**
     * Adds the location by [sic] another.
     *
     * @param vec The other location
     * @return the same location
     * @throws IllegalArgumentException for differing worlds
     */
    public Location add(Location vec) {
        if (vec == null || vec.getWorld() != getWorld()) {
            throw new IllegalArgumentException("Cannot add Locations of differing worlds");
        }

        x += vec.x;
        y += vec.y;
        z += vec.z;
        return this;
    }
\end{lstlisting}
\newpage
(Extra space for Q1 answers)
\newpage

%%   private void removeFromOutgoing(Plugin plugin) {
    %%     synchronized (outgoingLock) {
    %%         Set<String> channels = outgoingByPlugin.get(plugin);

    %%         if (channels != null) {
    %%             String[] toRemove = channels.toArray(new String[0]);

    %%             outgoingByPlugin.remove(plugin);

    %%             for (String channel : toRemove) {
    %%                 removeFromOutgoing(plugin, channel);
    %%             }
    %%         }
    %%     }
    %%   }

    %% public Location setDirection(Vector vector) {
    %%     /*
    %%      * Sin = Opp / Hyp
    %%      * Cos = Adj / Hyp
    %%      * Tan = Opp / Adj
    %%      *
    %%      * x = -Opp
    %%      * z = Adj
    %%      */
    %%     final double _2PI = 2 * Math.PI;
    %%     final double x = vector.getX();
    %%     final double z = vector.getZ();

    %%     if (x == 0 && z == 0) {
    %%         pitch = vector.getY() > 0 ? -90 : 90;
    %%         return this;
    %%     }

    %%     double theta = Math.atan2(-x, z);
    %%     yaw = (float) Math.toDegrees((theta + _2PI) % _2PI);

    %%     double x2 = NumberConversions.square(x);
    %%     double z2 = NumberConversions.square(z);
    %%     double xz = Math.sqrt(x2 + z2);
    %%     pitch = (float) Math.toDegrees(Math.atan(-vector.getY() / xz));

    %%     return this;
    %% }



\section{Branch and Statement Coverage [15 Marks]}

Here's the implementation of Java's \texttt{java.util.random.Random.nextInt} function. (5 Marks) Draw the control-flow graph
for \texttt{nextInt}. (5 Marks) Since the utility function \texttt{next()} returns pseudorandom bits,
discuss any difficulties that may arise in ensuring 100\% statement coverage for \texttt{nextInt()}.
How can you ensure 100\% statement coverage? 
(5 Marks) What about branch coverage? What are the difficulties and how can you ensure 100\% branch coverage?

\vspace*{1em}

Assume that you can change any part of {\tt Random}'s state and call \texttt{nextInt} how you'd like; you may not change \texttt{nextInt} itself.
\begin{lstlisting}[language=Java]
    // credit: java.util.Random

    public int nextInt(int n) {
        if (n<=0)
            throw new IllegalArgumentException("n must be positive");

        if ((n & -n) == n)  // i.e., n is a power of 2
            return (int)((n * (long)next(31)) >> 31);

        int bits, val;
        do {
            bits = next(31);
            val = bits % n;
        } while(bits - val + (n-1) < 0);
        return val;
    }
\end{lstlisting}

\newpage

%Give a set of test cases that achieve 100\% statement coverage on this
%function.

    %%   private AtomicLong seed;

    %%   protected int next(int bits) {
    %%     long oldseed, nextseed;
    %%     do {
    %%       oldseed = seed.get();
    %%       nextseed = (oldseed * multiplier + addend) & mask;
    %%     } while (!seed.attemptUpdate(oldseed, nextseed));
    %%     return (int)(nextseed >>> (48 - bits));
    %% }
(Extra space for Q2 answers)
\newpage
\section{Input Generation [10 Marks]}

You have a test suite which contains the following set of calls to a REST API (\url{https://reqres.in}, specifically).
I've put the POST and PUT payloads in braces after the URL.

\begin{lstlisting}
GET /api/users
GET /api/users/2
POST /api/users { "name": "plam", "job": "SE director" }
PUT /api/users/2 { "name": "plam", "commute": "bicycle" }
DELETE /api/users/2
POST /api/register { "email": "patrick.lam@uwaterloo.ca", "password": "password1" }
POST /api/login { "email": "patrick.lam@uwaterloo.ca", "password": "password2" }
POST /api/logout { "token": "QpwL5tke4Pnpja7X" }
\end{lstlisting}

(2 Marks) Provide an additional input that looks correct and an input that looks incorrect.
(6 Marks)~Provide pseudocode that programmatically generates correct inputs, including at least 4 of the calls above. Your pseudocode may call primitives that randomly generate an integer or string.
(2 Marks)~Describe how to programmatically generate incorrect inputs. (A good way of
describing is by modifying the pseudocode).

\newpage
\section{Finite State Machines [10 Marks]}
(4 Marks) Propose a Finite State Machine for the REST API in the previous question. The FSM should abstract away from the details in the
example requests. It should also include an appropriate cycle. (There are multiple possible correct answers). (6 Marks) Write down the test
requirements for Simple Round Trip Coverage and a test suite which satisfies these test requirements. Are there additional test
requirements for Complete Round Trip Coverage in your FSM?


%Here are some inputs. Provide a grammar and an additional input that conforms to the grammar.
%Then, give a second input that ``almost conforms'' to the grammar, in the sense that one change to the
%grammar suffices to generate your second input.

%--> sequence of API calls
%get list, retrieve thing, modify it, push thing


%%%%%%%%%%%%%%%%%%%%%%%%%%%%%%%%%%%%%%%%%%%%%%%%%%%%%%%%%%%%%%%%%%%%%%%%%%%%%%%
%% Some tests may want to end with blank pages for student's rough work, or
%% for more space to write answers.  Some optional instructions are given below.
%% \newpage 

%% \begin{center}
%% Extra page for answers.
%% Please specify the question number here and the use of this page on
%% the question page.
%% \end{center}

\newpage
\section{Short Answer [15 Marks total]}

Answer these questions using at most three sentences. Each question is worth
3 marks.

\begin{enumerate}[label=(\alph*)]

\item Give an example of a bug that is best detected by exploratory testing.
  \vspace*{6em}
\item Your test suite achieves 55\% statement coverage and the tests all pass. Without any further information, what is one fact that you can conclude about the statements that are reported as covered?
  \vspace*{6em}
\item For the same test suite as above, what can you conclude about the 45\% of statements that aren't covered?
  \vspace*{6em}
\item You are developing a test suite for a web page that is to be translated into multiple languages. How do you introduce a level of abstraction into your tests?
  \vspace*{6em}
\item Consider a reachable fault $F$ that infects the program state, propagating to output. Say you delete the line of code containing $F$. Would you still expect a failure? Where is the fault now?
  \vspace*{6em}

%\item Describe one idiom in the code which makes it difficult to achieve 100\% statement coverage.
%\item Web frontend testing

\end{enumerate}

\end{document}

